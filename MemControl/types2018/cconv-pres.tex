\documentclass{beamer}

\usetheme{default}
\useoutertheme{infolines}
\usepackage{amssymb}
\usepackage{amsmath}
\usepackage{hyperref}
\usepackage{proof}
\usepackage{tikz-cd}
\usepackage[utf8]{inputenc}
\usepackage{color}

\usetheme{Boadilla}
\setbeamertemplate{navigation symbols}{}
\setbeamertemplate{footline}[frame number]
\setbeamerfont{itemize/enumerate subbody}{size=\normalsize}
\setbeamerfont{itemize/enumerate subsubbody}{size=\normalsize}

% this file is needed to build telescopes.tex

%% %% setting \sqcdot as HoTT-book-style transitivity
%% \makeatletter
%% \DeclareRobustCommand{\sqcdot}{\mathbin{\mathpalette\morphic@sqcdot\relax}}
%% \newcommand{\morphic@sqcdot}[2]{%
%%   \sbox\z@{$\m@th#1\centerdot$}%
%%   \ht\z@=.33333\ht\z@
%%   \vcenter{\box\z@}%
%% }
%% \makeatother


\renewcommand{\U}{\mathsf{U}}
\newcommand{\El}{\mathsf{El}}
\newcommand{\REN}{\mathsf{REN}}
\newcommand{\op}{\mathsf{op}}
\newcommand{\ra}{\rightarrow}
\newcommand{\Ra}{\Rightarrow}

\newcommand{\Set}{\mathsf{Set}}
\newcommand{\PSh}{\mathsf{PSh}}
\newcommand{\FamPSh}{\mathsf{FamPSh}}
\renewcommand{\ll}{\llbracket}
\providecommand{\rr}{\rrbracket}
\newcommand{\Con}{\mathsf{Con}}
\newcommand{\Ty}{\mathsf{Ty}}
\newcommand{\Tm}{\mathsf{Tm}}
\newcommand{\Tms}{\mathsf{Tms}}
\newcommand{\R}{\mathsf{R}}
\newcommand{\TM}{\mathsf{TM}}
\newcommand{\NE}{\mathsf{NE}}
\newcommand{\NF}{\mathsf{NF}}
\newcommand{\p}{\mathsf{p}}
\newcommand{\q}{\mathsf{q}}
\renewcommand{\u}{\mathsf{u}}
\renewcommand{\ne}{\mathsf{ne}}
\newcommand{\nf}{\mathsf{nf}}
\newcommand{\lQ}{\mathsf{lQ}}
\newcommand{\lU}{\mathsf{lU}}
\renewcommand{\lq}{\mathsf{lq}}
\newcommand{\lu}{\mathsf{lu}}
\newcommand{\cul}{\ulcorner}
\newcommand{\cur}{\urcorner}
\newcommand{\norm}{\mathsf{norm}}
\newcommand{\Nf}{\mathsf{Nf}}
\newcommand{\Ne}{\mathsf{Ne}}
\newcommand{\Nfs}{\mathsf{Nfs}}
\newcommand{\Nes}{\mathsf{Nes}}
\newcommand{\ID}{\mathsf{ID}}
\newcommand{\id}{\mathsf{id}}
\newcommand{\nat}{\,\dot{\rightarrow}\,}
%\newcommand{\nat}{\overset{\mathsf{n}}{\ra}} % this is how we denote it in the formalisation
\newcommand{\Nat}{\mathsf{Nat}}
\renewcommand{\S}{\overset{\mathsf{s}}{\ra}} % we have it with uppercase S in the formalisation
\newcommand{\blank}{\mathord{\hspace{1pt}\text{--}\hspace{1pt}}} %from the book
%\newcommand{\blank}{\!{-}\!}
\newcommand{\lam}{\mathsf{lam}}
\newcommand{\app}{\mathsf{app}}
\newcommand{\tr}[2]{\ensuremath{{}_{#1 *}\mathopen{}{#2}\mathclose{}}}
\renewcommand{\C}{\mathsf{C}}
\newcommand{\Code}{\mathsf{Code}}
\renewcommand{\M}{\mathsf{M}}
% from the book
%\newcommand{\M}{{\scalebox{0.6}{$\mathsf{M}$}}}
%% \newcommand{\C}{\mathcal{C}}
\newcommand{\data}{\mathsf{data}}
\newcommand{\ind}{\hspace{1em}}
\newcommand{\idP}{\mathsf{idP}}
\newcommand{\compP}{\mathsf{compP}}
\newcommand{\idF}{\mathsf{idF}}
\newcommand{\compF}{\mathsf{compF}}
\newcommand{\proj}{\mathsf{proj}}
\newcommand{\ExpPSh}{\mathsf{ExpPSh}}
\newcommand{\map}{\mathsf{map}}
\newcommand{\Var}{\mathsf{Var}}
\newcommand{\Vars}{\mathsf{Vars}}
\newcommand{\wk}{\mathsf{wk}}
\newcommand{\neuU}{\mathsf{neuU}}
\newcommand{\neuEl}{\mathsf{neuEl}}
\newcommand{\var}{\mathsf{var}}
\newcommand{\natn}{\mathsf{natn}}
\newcommand{\natS}{\mathsf{natS}}
\newcommand{\LET}{\mathsf{let}}
\newcommand{\IN}{\mathsf{in}}
\newcommand{\refl}{\mathsf{refl}}
\newcommand{\trans}{\mathbin{\raisebox{0.2ex}{$\displaystyle\centerdot$}}}
\newcommand{\zero}{\mathsf{zero}}
\newcommand{\suc}{\mathsf{suc}}
\newcommand{\N}{\mathbb{N}}

\newcommand\arcfrombottom{
  \begin{tikzpicture}[scale=0.03em]
    \draw (0,0) arc (0:180:0.5);
    \draw (0,0) edge[->] (0,-0.3);
    \draw (-1,0) edge (-1,-0.3);
  \end{tikzpicture}
}
\newcommand\arcfromtop{
  \begin{tikzpicture}[scale=0.03em]
    \draw (0,0) arc (180:360:0.5);
    \draw (0,0) edge[->] (0,0.3);
    \draw (1,0) edge (1,0.3);
  \end{tikzpicture}
}

\newcommand{\Elim}{\mathsf{Elim}}
\newcommand{\elim}{\mathsf{elim}}
\newcommand{\Rec}{\mathsf{Rec}}
\newcommand{\record}{\mathsf{record}}
\newcommand{\funext}{\mathsf{funext}} \newcommand{\Q}{\mathsf{Q}}
\renewcommand{\T}{\mathsf{T}} \newcommand{\leaf}{\mathsf{leaf}}
\newcommand{\node}{\mathsf{node}} \newcommand{\perm}{\mathsf{perm}}
\newcommand{\coe}{\mathsf{coe}} \newcommand{\vz}{\mathsf{vz}}
\newcommand{\vs}{\mathsf{vs}} \newcommand{\untr}{\mathsf{untr}}
\newcommand{\from}{\mathsf{from}}
\newcommand{\fromeq}{{\mathsf{from}\hspace{-0.3em}\equiv}}
\newcommand{\fromsimeq}{{\mathsf{from}\hspace{-0.3em}\simeq}}
\newcommand{\Model}{\mathsf{Model}}
\newcommand{\DModel}{\mathsf{DModel}}
\newcommand{\module}{\mathsf{module}}
\newcommand{\open}{\mathsf{open}}
\renewcommand{\P}{\mathsf{P}} \newcommand{\Bool}{\mathsf{Bool}}
\newcommand{\true}{\mathsf{true}} \newcommand{\false}{\mathsf{false}}
\renewcommand{\not}{\mathsf{not}} \newcommand{\0}{\mathsf{0}}
\newcommand{\1}{\mathsf{1}} \renewcommand{\Pr}{\mathsf{Pr}}
\newcommand{\PrNat}{\mathsf{PrNat}} \newcommand{\J}{\mathsf{J}}
\newcommand{\wkV}{\mathsf{wkV}} \renewcommand{\r}[1]{{\P_{#1}}}
\newcommand{\stab}{\mathsf{stab}} \newcommand{\NTy}{\mathsf{NTy}}
\newcommand{\isDec}{\mathsf{isDec}} \newcommand{\dec}{\mathsf{dec}}
\newcommand{\yes}{\mathsf{yes}} \newcommand{\no}{\mathsf{no}}
\newcommand{\case}{\mathsf{case}} \newcommand{\inj}{\mathsf{inj}}
\newcommand{\K}{\mathsf{K}} \newcommand{\lb}{\langle}
\newcommand{\rb}{\rangle} \newcommand{\Decl}{\mathsf{Decl}}
\newcommand{\Core}{\mathsf{Core}} \newcommand{\IND}{\mathsf{ind}}
\newcommand{\Id}{\mathsf{Id}} \newcommand{\Base}{\mathsf{Base}}
\newcommand{\Setoid}{\mathsf{Setoid}}
\newcommand{\FamSetoid}{\mathsf{FamSetoid}}
\newcommand{\prop}{\mathsf{prop}} \newcommand{\resp}{\mathsf{resp}}
\newcommand{\transport}{\mathsf{transport}}
\newcommand{\I}{\mathsf{I}} \newcommand{\E}{\mathsf{E}}
\newcommand{\transp}{\mathsf{transp}}
\newcommand{\Transp}{\mathsf{Transp}}
\newcommand{\W}{\mathsf{W}}
\newcommand{\Fin}{\mathsf{Fin}} \newcommand{\fzero}{\mathsf{fzero}}
\newcommand{\fsuc}{\mathsf{fsuc}}
\newcommand{\inv}{\mathsf{inv}}
\newcommand{\con}{\mathsf{con}}
\newcommand{\LEFT}{\mathsf{left}}
\newcommand{\RIGHT}{\mathsf{right}}
\newcommand{\seg}{\mathsf{seg}}
\newcommand{\Int}{\mathsf{Int}}
\renewcommand{\in}{\mathbin{\hat:}}
%% \renewcommand{\hat}[1]{{\color{BrickRed}{#1}}}
\newcommand{\vdashh}{\mathbin{\hat\vdash}}
\newcommand{\rah}{\mathbin{\hat\ra}}
\newcommand{\commah}{\hat,\,}
\newcommand{\timesh}{\mathbin{\hat\times}}
\newcommand{\eqh}{\mathbin{\hat=}}
\newcommand{\TR}{\hat{\mathsf{tr}}}
\newcommand{\ap}{\hat{\mathsf{ap}}}
\newcommand{\apd}{\hat{\mathsf{apd}}}
\renewcommand{\tt}{\hat{\mathsf{tt}}}
\newcommand{\Tel}{\hat{\mathsf{Tel}}}
\newcommand{\Type}{\hat{\mathsf{Type}}}
\newcommand{\emptytel}{\hat{\epsilon}}
\newcommand{\telext}{\mathbin{\hat{\lhd}}}
\newcommand{\ltel}{\hat{\ll}}
\newcommand{\rtel}{\hat{\rr}}
\newcommand\pp{\ensuremath{\hat{\mathbin{+\mkern-10mu+}}}}
%% \newcommand{\semicol}{\hat;\,}
%% \newcommand{\targetass}{\hat{\Gamma}\semicol}
%\newcommand{\targetass}{;}

\newcommand{\type}{\mathsf{type}}
\newcommand{\Cl}{\mathsf{Cl}}
\newcommand{\pack}{\mathsf{pack}}
\newcommand{\level}{\mathsf{level}}
\newcommand{\open}{\mathsf{open}}
\newcommand{\close}{\mathsf{close}}

%% \AtBeginSection[]{
%%   \begin{frame}
%%   \vfill
%%   \centering
%%   \begin{beamercolorbox}[sep=8pt,center,shadow=true,rounded=true]{title}
%%     \usebeamerfont{title}\insertsectionhead\par%
%%   \end{beamercolorbox}
%%   \vfill
%%   \end{frame}
%% }

\title{Closure Conversion for Dependent Type Theory}
\subtitle{With Type-Passing Polymorphism\thanks{This work was supported by the European Union, co-financed by the European
Social Fund (EFOP-3.6.3-VEKOP-16-2017-00002).}}
\author{András Kovács}
\institute{Eötvös Loránd University, Budapest}
\date{TYPES 2018, 20 June 2018}


\begin{document}

\frame{\titlepage}


\begin{frame}{Advertising}


William J. Bowman and Amal Ahmed: \emph{Typed Closure Conversion for the Calculus of Constructions}, PLDI 2018, Philadelphia.

\begin{itemize}
\item
Significant overlap with the current talk. I was unaware of the
preprint until a kind TYPES reviewer pointed it out to me.
\item
The basic technical idea (abstract closures) is the same as here (independent validation!).
\item
I encourage interested people to read this paper for details.
\end{itemize}

\end{frame}

\begin{frame}{Motivation}
  \begin{itemize}
  \item
    Variants of dependent type theory proliferate: quantitative, cubical, guarded, etc.
  \item
    We would like to add: \emph{type theory with precise memory layout control.}
    \begin{itemize} \item Basic example: $\Sigma$ interpreted as (dependent) sequential memory layout.
    \end{itemize}
  \item
    Hopefully eventually complementing the resource usage control of quantitative
    type theories.
  \item Benefits:
    \begin{itemize}
    \item As front-end language: more control for programmers.
    \item As intermediate language: well-typed transformations, general handling of memory layout.
    \end{itemize}
  \end{itemize}
\end{frame}

\begin{frame}{Ingredients of memory layout control}

  We need to make some new distinctions:

  \begin{itemize}
  \item Types vs. runtime type codes
  \item Closed functions vs. closures
  \item Consecutive layout vs. pointers
  \item Uniform vs. variable sized data
  \item Alignment
  \item (more things)
  \end{itemize}
(Also: lots of required further research \& work down the compilation pipeline)
\end{frame}

\begin{frame}{Ingredients of memory layout control}

  We need to make some new distinctions:

  \begin{itemize}
  \item {\color{red} Types vs. runtime type codes}
  \item {\color{red} Closed functions vs. closures}
  \item Consecutive layout vs. pointers
  \item Uniform vs. variable sized data
  \item Alignment
  \item (more things)
  \end{itemize}
(Also: lots of required further research \& work down the compilation pipeline)
\end{frame}

\begin{frame}{Current contribution}

  A small type theory where:
  \begin{itemize}
  \item There aren't general dependent functions, only closed functions and closures.
  \item But general dependent functions remain admissible, through closure conversion.
  \item Type codes also use closures to represent type dependency.
  \item Consistency follows from a straightforward syntactic translation to closure-free
        MLTT.
  \end{itemize}
\end{frame}


\begin{frame}{Type-passing polymorphism}
Why have closures in type codes?

  \begin{itemize}
  \item
  This allows efficient layout computation at runtime.
  \item
  For example: computing the size of a value with $\Sigma$-type.
  \item
  See: \emph{Harper \& Morrisett: Compiling Polymorphism Using Intensional Type Analysis}.
  \item
    Intensional (synonymously: type-passing) polymorphism generalizes type erasure (e. g. GHC Haskell) and monomorphization (e. g. Rust, C++).
  \item
  We don't want to \emph{rule out} type-passing polymorphism down the compilation pipeline.
  \end{itemize}

\end{frame}

\begin{frame}{The type theory (1)}

Judgements:
\begin{gather*}
  \Gamma \vdash  \hspace{2em}  \Gamma \vdash A\,\type_i \hspace{2em} \Gamma \vdash t : A
\end{gather*}
Universes:
\begin{gather*}
  \infer{\Gamma \vdash \U_i\,\,\type_{i+1}}{}
  \hspace{2em}
  \infer{\Gamma \vdash \El\,A : \type_i}{\Gamma \vdash A : \U_i}
\end{gather*}
Closed functions:
\begin{gather*}
  \infer{\Gamma \vdash (a : A)\ra B\,\,\type_{\max(i,\,j)}}{\Gamma \vdash A\,\type_i & \Gamma, a : A \vdash B\,\type_j}
  \hspace{1em}
  \infer{\Gamma \vdash \lambda a.\,t : (a : A)\ra B}{\boldsymbol{\cdot},\,a : A \vdash t : B}
\end{gather*}
\begin{itemize}
\item Standard application, $\beta$ and $\eta$ for closed functions.
\item Standard $\Sigma$ types and $\top$ (unit type).
\end{itemize}
\end{frame}

\begin{frame}{}

  Closed functions are quite restricted.\\~\\
  The usual polymorphic identity function isn't possible: $\lambda A.\,\lambda (x : {\color{red}\El\,A}).\, x$.\\~\\
  Instead, we may have $\lambda(A,\,x).\,x : (x : \Sigma(A : U).\,\El\,A) \ra \El\,(\proj_1\,x)$.

\end{frame}

\begin{frame}{Closures}

\begin{gather*}
  \infer{\Gamma \vdash \Cl\,(a : A)\,B\,\,\type_{\max(i,\,j)}}{\Gamma \vdash A\,\type_i & \Gamma, a : A \vdash B\,\type_j}
\end{gather*}
\begin{gather*}
\infer{\Gamma \vdash \pack\,E\,env\,\,t : \Cl\,(a : A[e\mapsto env])\,(B[ea \mapsto (env,\,a)])}{\boldsymbol{\cdot} \vdash E : \U_i & \Gamma \vdash env : \El\, E & \boldsymbol{\cdot} \vdash t : (ea : \Sigma(e : \El\,E).A) \ra B}
\end{gather*}
\begin{gather*}
  \infer{\Gamma \vdash t\,u : B[a \mapsto u]}{\Gamma \vdash t : \Cl\,(a : A)\,B & \Gamma \vdash u : A}
\end{gather*}
\begin{gather*}
  \infer{\Gamma \vdash t \equiv u}{\Gamma \vdash t : \Cl\,(a : A)\,B & \Gamma \vdash u : \Cl\,(a : A)\,B & \Gamma,\,a : A \vdash t\,a \equiv u\,a}
\end{gather*}
\begin{gather*}
  (\pack\,E\,env\,\,t)\,u \equiv t\,(env,\,u)
\end{gather*}
\end{frame}

\begin{frame}{Type codes}

Universe:

\begin{gather*}
\infer{\Gamma \vdash \U'_i : \U_{i+1}}{}\hspace{2em} \El\,\U'_i\equiv \U_i
\end{gather*}

Codes for $\Cl$:

\begin{gather*}
  \infer{\Gamma \vdash \mathsf{Cl'}\,A\,B : \U_{\max(i,\,j)}}
        {\Gamma \vdash A : \U_i & \Gamma \vdash B : \Cl\,(\El\,A)\,(\U_j)}
  \hspace{1.5em}
  \El\,(\mathsf{Cl'}\,A\,B)\equiv \Cl\,(a : \El\,A)\,(\El\,(B\,a))
\end{gather*}
\\~\\
Analogously for $\Sigma$, $\top$ and closed functions.

\end{frame}

\begin{frame}
\\~\\
\\~\\
Polymorphic identity function with closures:
  \begin{alignat*}{5}
    & \mathsf{id} : \Cl(A : \U)(\Cl(x : \El\,A)(\El\,A)) \\
    & \mathsf{id} :\equiv \pack\,\top'\,\tt\,(\lambda(\tt,\,A).\, \pack\,\U'\,A\,(\lambda(A,\,x).\,x))
\end{alignat*}
\end{frame}

\begin{frame}{Closure conversion}
To show: general closure abstraction, notated here as $\lambda\{x\}.\,t$, is admissible.

\begin{gather*}
\infer{\Gamma \vdash \lambda\{a\}.\,t : \Cl\,(a : A)\,B}{\Gamma,\,a : A \vdash t : B}  \hspace{2em}  \lambda\{x\}.\,t\,x\equiv t \hspace{2em} (\lambda\{x\}.\,t)\,u \equiv t[x \mapsto u]
\end{gather*}

  $\lambda\{x\}.\,t$ is given mutually with a number of operations, which are given by mutual induction on contexts and types.\\~\\

$\Gamma \vdash \sigma : \Delta$ will denote a parallel substitution, $\mathsf{id}$ identity substitution, $\circ$ composition.\\~\\

\end{frame}

\begin{frame}{Induction motive for contexts}
\begin{gather*}
  \infer
      {\begin{array}{l}
          \level\,\Gamma \in \mathbb{N}\\
          \boldsymbol{\cdot} \vdash \mathsf{quote}\,\Gamma : \U_{(\level\,\Gamma)}\\
          \Gamma \vdash \open\,\Gamma : \El\,(\mathsf{quote}\,\Gamma)\\
          e : \El\,(\mathsf{quote}\,\Gamma) \vdash \close\,\Gamma : \Gamma\\
          $[e \mapsto \open\,\Gamma\,[ \close\,\Gamma ] ] \equiv \id$\\
          $\close\,\Gamma\circ[e \mapsto \open\,\Gamma]\equiv \id$
      \end{array}}
    {\Gamma \vdash}
\end{gather*}
  \begin{itemize}
    \item
      $\mathsf{quote}$ converts $\Gamma$ to a code for an iterated left-nested $\Sigma$-type.
    \item $\open\,\Gamma$ fills such a $\Sigma$ with variables from the context, for example: $\open\,(x : \U, y : \U) \equiv ((tt,\,x),\,y)$.
    \item $\close\,\Gamma$ is a substitution which converts variables of $\Gamma$ to projections from $e : \mathsf{quote}\,\Gamma$, for example: $(x,\,y)[\close\,(x : \U, y : \U)] \equiv (\proj_2 (\proj_1 e),\, \proj_2 e)$.
  \end{itemize}
\end{frame}

\begin{frame}{Induction motive for types, closure building}

Induction motive for types:
\begin{gather*}
  \infer
      {\begin{array}{l}
          \Gamma \vdash \mathsf{quote}\,A : \U_i\\
          \Gamma \vdash \El\,(\mathsf{quote}\,A) \equiv A\\
          $\forall \sigma.\, \mathsf{quote}\,A\,[\sigma]
            \equiv \mathsf{quote}\,(A\,[\sigma])$
      \end{array}}
    {\Gamma \vdash A\,\type_i}
\end{gather*}
Closure building:
\begin{gather*}
  \infer
      {\lambda\{a\}.\,t :\equiv \pack\,(\mathsf{quote}\,\Gamma)\,(\open\,\Gamma)\,
        (\lambda e.\, t\, [\close\,(\Gamma,\,a : A)])}
      {\Gamma,\, a : A \vdash t : B}
\end{gather*}
\end{frame}

\begin{frame}
\begin{center} {\large Thank you!} \end{center}
\end{frame}

\end{document}
